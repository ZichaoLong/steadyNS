\subsection{Data Structure}
Simplex grid partition for the computation domain $\Omega$ is adopted, 
and without damage to the functional setting of Eq.\eqref{eq:Navier-Stokes}, 
we will use periodic boundary conditions at some boundary surfaces of $\Omega$.
The data structure reads as follows:
\begin{itemize}
  \item $N$ nodes: $\{A_i\in\mr^d:i=0,\cdots,N-1\}$ and 
    $NE$ edges $\{A_i\in\mr^d:i=N,\cdots,N+NE-1\}$ 
    with indices $0,\cdots,N-1$ and $N,\cdots,N+NE-1$ respectively.

  \item Node tag: $B=\{B_i:i=0,\cdots,N+NE-1\}$, which defines the physical information of $i$-th point of degree of freedom.
    \begin{description}
      \item[0] interior node
      \item[1] inlet, where $u=(1,0,0)^\top$
      \item[2] outlet, where $u=(1,0,0)^\top$
      \item[3] noslip wall, where $u=(0,0,0)^\top$. 
        Velocity value $u$ on the surface of cylinders/spheres can be changed 
        if angular velocity $\omega\neq0$. Here we only consider flow past still cylinders/spheres.
      \item[4] periodic node
      \item[-1] periodic source node
    \end{description}
    Only those nodes with tag $B_i=0$ or $B_i=-1$ have degrees of freedom.

  \item Periodic points indices: $P=\{P_i\in\{-1,0,1,\cdots,N+NE-1\}\}_{i=0}^{N+NE-1}$.
    \[
      P_i=\left\{\begin{array}{ll}
        j, & A_j \text{ is the source node of } A_i \\
        -1, & \text{ else }
      \end{array}\right.
      \]
    which satisfies $P_i=-1\forall i\in \{P_j\}$, 
    i.e. each source node do not have source node.

  \item $M$ elements: $\{e_k:k=0,\cdots,M-1\}\subset\{0,1,\cdots,N+NE-1\}^{(d+1)(d+2)/2}$. 
    Each simplex element $e_k$, e.g. triangle or tetrahedron, has $d+1$ nodes $\{e_k[j]\}_{j=0}^{d}$. 
    In the following report, indices of nodes of an element will be supercripts.
    $e_k[d+1:]$ is the $d(d+1)/2$ edges of $e_k$:
    \begin{itemize}
      \item if $d=2$, then $e_k[d+1+j]$ is the edge opposite to $e_k[j]$
      \item if $d=3$, then $e_k$ is the six edges in the order of $(1,2),(1,3),(1,4),(2,3),(2,4),(3,4)$
    \end{itemize}

\end{itemize}
In the following report, we use $i\in e_k$ and $i\sim j$ to represent the 
relation of containing and neighbouring respectively, i.e.
the $i\in\{e_k^j:j=0,\cdots,(d+1)(d+2)/2\}$ and 
we say $i\sim j$ if and only if there exists an element $e_k$ $\st i,j\in e_k$.

