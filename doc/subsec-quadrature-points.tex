\subsection{Quadrature Points}
A material fact is that, the integral of a polynomial can be easily calculated, 
in the sense of up to a multiplycative constant $|e|$,
by sum up the function values at quadrature points\cite{chen2008ifem}.
Thus, given the propriate quadrature points and the corresponding 
barycentric coordinate $\Lambda$ and weight $W$
(for example, see Tab.\ref{tab:quadrature-points-2d} and Tab.\ref{tab:quadrature-points-3d}),
the integral of an n-th order polynomial $g(x)$ can be writen as 
\begin{equation}\label{eq:quadrature-points-integral}
  \int_e g(x)\ud x=|e|\sum_{i=0}^{\text{nQuad}}W_n[i]g(\sum_{j=0}^{d}\lambda_n[i,j]e^j)
\end{equation}
\begin{table}[ht]
  \centering
  \caption{2D-case: Quadrature Points for polynomials with Order 1,2, 
  where $\Lambda[i,j]$ is $j$-th barycentric coordinate of $i$-th quadrature point equipment with a weight $W_i$. 
  In this report, only Order 2 will be used.}
  \label{tab:quadrature-points-2d}
  \begin{tabular}{l|l|l}
    \hline
    Order 1, nQuad 1
    & $W=1$
    & $\lambda=[1/3,1/3/,1/3]$ \\
    \hline
    Order 2, nQuad 3 
    & $W=\left[
      \begin{array}{c}
        1/3 \\
        1/3 \\
        1/3
      \end{array}\right]$
    & $\lambda=\left[
      \begin{array}{ccc}
        2/3, & 1/6, & 1/6 \\
        1/6, & 2/3, & 1/6 \\
        1/6, & 1/6, & 2/3
      \end{array}
      \right]$ \\
    \hline
    %    Order 3, nQuad 4
    %    & $W = \left[
    %      \begin{array}{c}
    %        -27/48\\
    %        25/48 \\
    %        25/48 \\
    %        25/48
    %      \end{array}\right]$
    %    & $\lambda=\left[
    %      \begin{array}{ccc}
    %        1/3,& 1/3,& 1/3 \\
    %        0.6,& 0.2,& 0.2 \\
    %        0.2,& 0.6,& 0.2 \\
    %        0.2,& 0.2,& 0.6 
    %      \end{array}\right]$ \\
    %    \hline
  \end{tabular}
\end{table}
\begin{table}[ht]
  \centering
  \caption{3D-case: Quadrature Points for polynomials with Order 1,2,
  where $\Lambda[i,j]$ is $j$-th barycentric coordinate of $i$-th quadrature point equipment with a weight $W_i$,
  The parameters for the case of Order 2 are set as follows:$\alpha = 0.5854101966249685; \beta = 0.138196601125015 $.
  In this report, only Order 2 will be used.}
  \label{tab:quadrature-points-3d}
  \begin{tabular}{l|l|l}
    \hline
    Order 1, nQuad 1
    & $W = 1$
    & $\lambda = [1/4, 1/4, 1/4, 1/4]$ \\
    \hline
    Order 2, nQuad 4
    & $W=\left[
      \begin{array}{c}
        1/4 \\
        1/4 \\
        1/4 \\ 
        1/4
      \end{array}\right]$
    &$\lambda=\left[\begin{array}{cccc}
      \alpha & \beta  & \beta  & \beta \\
       \beta  & \alpha & \beta  & \beta \\
       \beta  & \beta  & \alpha & \beta \\
       \beta  & \beta  & \beta  & \alpha 
    \end{array}\right]$ \\
     \hline
     %    Order 3, nQuad 5
     %    & $W=\left[
     %      \begin{array}{c}
     %        -4/5 \\
     %        9/20 \\
     %        9/20 \\
     %        9/20 \\
     %        9/20 
     %      \end{array}\right]$
     %    & $\lambda = \left[\begin{array}{cccc}
     %      1/4 & 1/4 & 1/4 & 1/4 \\
     %      1/2 & 1/6 & 1/6 & 1/6 \\
     %      1/6 & 1/2 & 1/6 & 1/6 \\
     %      1/6 & 1/6 & 1/2 & 1/6 \\
     %      1/6 & 1/6 & 1/6 & 1/2
     %    \end{array}\right]$ \\
     %    \hline
  \end{tabular}
\end{table}

