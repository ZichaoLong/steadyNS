\subsection{Discrete Space}
\textbf{Velocity Field $u$} is discreted by P2 element, 
with 1 degrees of freedom per node and 1 degrees of freedom per edge.
The discrete function space $V_h\subset V$ can be writen as
\begin{equation}\label{eq:u-space}
  V_h=\{\alpha_0v_0+\alpha_1v_1+\cdots+\alpha_{N+NE-1}v_{N+NE-1}:\alpha_i\}
\end{equation}
subject to that, for all $i\in\{0,\cdots,N+NE-1\}$
\begin{equation}\label{eq:u-constraint}
  \alpha_i=\left\{
  \begin{array}{ll}
    (1,0,0)^\top \text{ or } (1,0)^\top & \text{if } 1\leq B_i\leq 3 \\
    (0,0,0)^\top \text{ or } (0,0)^\top & \text{if } B_i=4
  \end{array}
  \right.
\end{equation}
where $\alpha_i\in\mr^d$ and $v_i:\Omega\to\mr$ is a 
quadratic function such that $v_i(A_i)=1,v_i(A_j)=0,\forall j\neq i$.
To construct the space of test function $v$, we would simply change the first condition in 
Eq.\eqref{eq:u-constraint} "$\alpha_i=(1,0,0)^\top \text{ or } (1,0)^\top \text{ if } 1\leq B_i\leq 3$" to 
"$\alpha_i=(0,0,0)^\top \text{ or } (0,0)^\top \text{ if } 1\leq B_i\leq3$".

\textbf{Pressure $p$}, at the same time, is approximated by P1 element 
with 1 degrees of freedom per node. Therefore, the space $Q_h\subset Q$ reads as
\begin{equation}\label{eq:p-space}
  Q_h=\{\beta_0q_0+\beta_1q_1+\cdots+\beta_{N-1}q_{N-1}:\beta_i\}
\end{equation}
subject to $\int\sum_{k=0}^{N-1}\beta_kq_k=0$, where $\beta_k\in\mr$ and $q_k:\Omega\to\mr$,
The space of test function $q$ is the same as $p$.

