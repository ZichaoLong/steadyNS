\subsection{Basis Function}
Before building up the linear system of solution variables $\{\alpha_i^n\}_{i=0}^{N+NE-1}$ and $\{\beta_i^n\}_{i=0}^{M-1}$,
we should introduce the close form of integral of polynomials,
in a single triangle/tetrahedron element $e$. 
This would be done by means of barycentric coordinate and quadrature points.

\paragraph{Barycentric Coordinate}
The barycentric coordinate 
$\lambda(x)=[\lambda^0,\cdots,\lambda^d]^\top\in\mr^{(d+1)\times1}$, 
which are also linear functions of $x$-$y$-$z$ (or $x$-$y$) coordinate 
$x\in\mr^{d\times1}$, can be calculated by applying the following linear transform:
\[
  \lambda(x)=
  \left[\begin{array}{c}
    1 \\
    e
  \end{array}
  \right]^{-1}
  \left[\begin{array}{c}
    1 \\
    x
  \end{array}\right]
\]
where the transformation matrix, in which $e^j\in\mr^{d\times1}$ stands for the 
$x$-$y$-$z$ (or $x$-$y$) coordinate of 
$j$-th$(j=0,\cdots,d)$ nodes of $e$, reads as follows:
\[
  2D:
  \left[\begin{array}{c}
    1 \\
    e
  \end{array}
  \right]=
\left[\begin{array}{ccc}
  1   & 1   & 1   \\
  e^1 & e^2 & e^3 \\
\end{array}
  \right]
  , \text{ or for }3D:
  \left[\begin{array}{c}
    1 \\
    e
  \end{array}
  \right]=
\left[\begin{array}{cccc}
  1   & 1   & 1   & 1   \\
  e^1 & e^2 & e^3 & e^4 \\
\end{array}
  \right]
\]
Soon we deduce that the measure of $e$, i.e. area for $2D$ case and volumn for $3D$ case, 
is $|e|=\frac{1}{d!}|\det([1;e])|$.
It is important to introduce the gradient of $\lambda\in\mr^{(d+1)\times1}$
\[
  \nabla \lambda = E^e
  \triangleq\left(\left[\begin{array}{c} 1 \\ e \end{array}\right]^{-1}\right)[:,1\text{:end}]
    \]
  As a simple example to illustrate barycentric coordinate, 
we consider linear function and quadratic function in $e$.

\paragraph{P1: linear function} $g:e\to\mr$, and then
\[g(x)=g({\textstyle\sum_{j=0}^{d}}\lambda^j(x)e^j)=\sum_{j}\lambda^j(x)g(e^j).\]
That is to say, $\{\lambda^j\}$ is the basis functions of piecewise linear element in element $e$. 
Rewrite $g(x)$ as $g=[g(e^j)]_{j=0}^d\in\mr^{1\times(d+1)}$, then 
\[g(x)=g\cdot\lambda(x),\nabla g(x)=g\cdot E^e\in\mr^{1\times d}\]
Specifically, for $g=\lambda^j$ we would have $\lambda^j(\sum_{j=0}^{d}\lambda^j(x)e^j)=\lambda^j(x)$.

\paragraph{P2: quadratic function} $g:e\to\mr$, 
$g$ is determined uniquely by $(d+1)(d+2)/2$ points in 
$e\in\mn^{(d+1)(d+2)/2}$, 
\[g(x)=\sum_{j=0}^dg(e^j)2\lambda^j(x)(\lambda^j(x)-\frac12)+
\sum_{j_2\leq j_1}g\big(\frac{e^{j_1}+e^{j_2}}2\big)4\lambda^{j_1}(x)\lambda^{j_2}(x)\]
and for convenience of notations we rewrite $g$ as $g\in\mr^{(d+1)(d+2)/2}$, 
\begin{equation}\label{eq:quadratic-freedom}
  g[j_1,j_2]=g[j_1(j_1+1)/2+j_2]=g(\frac{e^{j_1}+e^{j_2}}2), 0\leq j_2\leq j_1\leq d
\end{equation}
and $\lambda_2$ the basis function of P2 element:
\[
  \lambda_2^{j_1(j_1+1)/2+j_2}=\lambda_2^{j_1,j_2}=\left\{
  \begin{array}{ll}
    2\lambda^j(\lambda^j-\frac12) & \text{if } j_1=j_2=j \\
    4\lambda^{j_1}\lambda^{j_2}   & \text{if } j_2<j_1
  \end{array}
  \right., 0\leq j_2\leq j_1\leq d
\]
thus we have
\[g(x)=g\cdot\lambda_2(x)\]
the gradients of the basis funciton can be written as
\[
  \nabla\lambda_2^{j_1(j_1+1)/2+j_2}=\nabla\lambda_2^{j_1,j_2}=\left\{
  \begin{array}{ll}
    4\lambda^jE^e[j]-E^e[j]                        & \text{if } j_1=j_2=j \\
    4(\lambda^{j_1}E^e[j_2]+\lambda^{j_2}E^e[j_1]) & \text{if } j_2<j_1
  \end{array}
  \right., 0\leq j_2\leq j_1\leq d
\]
