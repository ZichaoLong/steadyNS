\subsection{Basis Function}
Before building up the linear system of solution variables $\{\alpha_i^n\}_{i=0}^{N-1}$ and $\{\beta_i^n\}_{i=0}^{M-1}$,
we should introduce the close form of integral of polynomials,
in a single triangle/tetrahedron element $e$. 
This would be done by means of barycentric coordinate and quadrature points.

\paragraph{Barycentric Coordinate}
The barycentric coordinate 
$\lambda(x)=[\lambda^0,\cdots,\lambda^d]^\top\in\mr^{(d+1)\times1}$, 
which are also linear functions of $x$-$y$-$z$ (or $x$-$y$) coordinate 
$x\in\mr^{d\times1}$, can be calculated by applying the following linear transform:
\[
  \lambda(x)=
  \left[\begin{array}{c}
    1 \\
    e
  \end{array}
  \right]^{-1}
  \left[\begin{array}{c}
  1 \\
  x
\end{array}\right]
\]
where the transformation matrix, in which $e^j\in\mr^{d\times1}$ stands for the 
$x$-$y$-$z$ (or $x$-$y$) coordinate of 
$j$-th$(j=0,\cdots,d)$ nodes of $e$, reads as follows:
\[
  2D:
  \left[\begin{array}{c}
    1 \\
    e
  \end{array}
  \right]=
\left[\begin{array}{ccc}
  1   & 1   & 1   \\
  e^1 & e^2 & e^3 \\
\end{array}
  \right]
  , \text{ or for }3D:
  \left[\begin{array}{c}
    1 \\
    e
  \end{array}
  \right]=
\left[\begin{array}{cccc}
  1   & 1   & 1   & 1   \\
  e^1 & e^2 & e^3 & e^4 \\
\end{array}
  \right]
\]
Soon we deduce that the measure of $e$, i.e. area for $2D$ case and volumn for $3D$ case, 
is $|e|=\frac{1}{d!}|\det([1;e])|$.
As a simple example to illustrate barycentric coordinate, 
consider a \textbf{linear function} $g:e\to\mr$, and then
\[g(x)=g({\textstyle\sum_{j=0}^{d}}\lambda^j(x)e^j)=\sum_{j}\lambda^j(x)g(e^j).\]
That is to say, $\{\lambda^j\}$ is the basis functions of piecewise linear element in element $e$.
Specifically, for $g=\lambda^j$ we would have $\lambda^j(\sum_{j=0}^{d}\lambda^j(x)e^j)=\lambda^j(x)$.

The gradient of $g$, which is constant in $e$, is given by
\[\mr^{1\times d}\owns\nabla g=\sum_{j}g(e^j)\nabla(\lambda^j(x))
\]
where $\nabla \lambda^j$ depends only on the geometry of $e$:
\begin{equation}\label{eq:nabla-barycentric-coordinate}
  \mr^{1\times d}\owns\nabla(\lambda^j(x))=\nabla(E[j]\cdot x)=E[j],
  E\triangleq\left(\left[\begin{array}{c} 1 \\ e \end{array}\right]^{-1}\right)[:,1\text{:end}]
\end{equation}

\paragraph{Quadrature Points}
A material fact is that, the integral of a polynomial can be easily calculated, 
in the sense of up to a multiplycative constant $|e|$,
by sum up the function values at quadrature points\cite{chen2008ifem}.
Thus, given the propriate quadrature points and the corresponding 
barycentric coordinate $\lambda$ and weight $W$
(see Tab.\ref{tab:quadrature-points-2d} and Tab.\ref{tab:quadrature-points-3d}),
the integral of an n-th order(n=1,2,3) polynomial $g(x)$ can be writen as 
\begin{equation}\label{eq:quadrature-points-integral}
  \int_e g(x)\ud x=|e|\sum_{i=0}^{\text{nQuad}}W_ig(\sum_{j=0}^{d}\lambda_i^je^j)
\end{equation}
\begin{table}[ht]
  \centering
  \caption{2D-case: Quadrature Points for polynomials with Order 1,2,3, 
  where $\lambda_i^j$ is $j$-th barycentric coordinate of $i$-th quadrature point equipment with a weight $W_i$. 
  In this report, only Order 2 will be used.}
  \label{tab:quadrature-points-2d}
  \begin{tabular}{l|l|l}
    \hline
    Order 1, nQuad 1
    & $W=1$
    & $\lambda=[1/3,1/3/,1/3]$ \\
    \hline
    Order 2, nQuad 3 
    & $W=\left[
      \begin{array}{c}
        1/3 \\
        1/3 \\
        1/3
      \end{array}\right]$
    & $\lambda=\left[
      \begin{array}{ccc}
2/3, & 1/6, & 1/6 \\
1/6, & 2/3, & 1/6 \\
1/6, & 1/6, & 2/3
      \end{array}
      \right]$ \\
    \hline
%    Order 3, nQuad 4
%    & $W = \left[
%      \begin{array}{c}
%        -27/48\\
%        25/48 \\
%        25/48 \\
%        25/48
%      \end{array}\right]$
%    & $\lambda=\left[
%      \begin{array}{ccc}
%        1/3,& 1/3,& 1/3 \\
%        0.6,& 0.2,& 0.2 \\
%        0.2,& 0.6,& 0.2 \\
%        0.2,& 0.2,& 0.6 
%      \end{array}\right]$ \\
%    \hline
  \end{tabular}
\end{table}
\begin{table}[ht]
  \centering
  \caption{3D-case: Quadrature Points for polynomials with Order 1,2,3,
  where $\lambda_i^j$ is $j$-th barycentric coordinate of $i$-th quadrature point equipment with a weight $W_i$,
  The parameters for the case of Order 2 are set as follows:$\alpha = 0.5854101966249685; \beta = 0.138196601125015 $.
  In this report, only Order 2 will be used.}
  \label{tab:quadrature-points-3d}
  \begin{tabular}{l|l|l}
    \hline
    Order 1, nQuad 1
    & $W = 1$
    & $\lambda = [1/4, 1/4, 1/4, 1/4]$ \\
    \hline
    Order 2, nQuad 4
    & $W=\left[
      \begin{array}{c}
        1/4 \\
        1/4 \\
        1/4 \\ 
        1/4
      \end{array}\right]$
    &$\lambda=\left[\begin{array}{cccc}
       \alpha & \beta  & \beta  & \beta \\
       \beta  & \alpha & \beta  & \beta \\
       \beta  & \beta  & \alpha & \beta \\
       \beta  & \beta  & \beta  & \alpha 
     \end{array}\right]$ \\
     \hline
%    Order 3, nQuad 5
%    & $W=\left[
%      \begin{array}{c}
%        -4/5 \\
%        9/20 \\
%        9/20 \\
%        9/20 \\
%        9/20 
%      \end{array}\right]$
%    & $\lambda = \left[\begin{array}{cccc}
%      1/4 & 1/4 & 1/4 & 1/4 \\
%      1/2 & 1/6 & 1/6 & 1/6 \\
%      1/6 & 1/2 & 1/6 & 1/6 \\
%      1/6 & 1/6 & 1/2 & 1/6 \\
%      1/6 & 1/6 & 1/6 & 1/2
%    \end{array}\right]$ \\
%    \hline
  \end{tabular}
\end{table}

